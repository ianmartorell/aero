\documentclass[10pt, a4paper]{article}
\usepackage{fontspec}
\usepackage{textcomp}
\usepackage{csquotes}
\usepackage[spanish]{babel}
\usepackage{authblk}
\usepackage{float}
\usepackage{pgfplots}
\usepackage{pgfplotstable}
\usepackage{graphicx}
\usepackage{gensymb}
\usepackage{listings}
\usepackage{color}

% Custom margins
\marginparwidth 0.5in
\oddsidemargin 0.25in
\evensidemargin 0.25in
\marginparsep 0.25in
\addtolength{\topmargin}{-.5in}
\textwidth 6in \textheight 9in
\linespread{1}
\setlength{\parindent}{0pt}
\setlength{\parskip}{3pt}

% Custom title
\makeatletter
\def\@maketitle{
  \newpage
  \null
  \vskip 2em
  \begin{center}
  \let \footnote \thanks
    {\Large\bfseries \@title \par}
    \vskip 1.5em
    {\normalsize
      \lineskip .5em
      \begin{tabular}[t]{c}
        \@author
      \end{tabular}\par}
    \vskip 1em%
    {\normalsize \@date}
  \end{center}
  \par
  \vskip 1.5em}
\makeatother

% pgfplots config
\pgfplotsset{compat=newest,every axis legend/.append style={at={(1.15,0)}, anchor=south},cycle list name=exotic}

% Code blocks formatting
\lstset{
  backgroundcolor=\color{white},   % choose the background color; you must add \usepackage{color} or \usepackage{xcolor}; should come as last argument
  basicstyle=\footnotesize,        % the size of the fonts that are used for the code
  breakatwhitespace=false,         % sets if automatic breaks should only happen at whitespace
  breaklines=true,                 % sets automatic line breaking
  captionpos=b,                    % sets the caption-position to bottom
  commentstyle=\color{teal},       % comment style
  deletekeywords={...},            % if you want to delete keywords from the given language
  escapeinside={\%*}{*)},          % if you want to add LaTeX within your code
  extendedchars=true,              % lets you use non-ASCII characters; for 8-bits encodings only, does not work with UTF-8
  frame=single,	                   % adds a frame around the code
  keepspaces=true,                 % keeps spaces in text, useful for keeping indentation of code (possibly needs columns=flexible)
  keywordstyle=\color{blue},       % keyword style
  language=Octave,                 % the language of the code
  morekeywords={*,...},            % if you want to add more keywords to the set
  numbers=left,                    % where to put the line-numbers; possible values are (none, left, right)
  numbersep=5pt,                   % how far the line-numbers are from the code
  numberstyle=\tiny\color{gray},   % the style that is used for the line-numbers
  rulecolor=\color{black},         % if not set, the frame-color may be changed on line-breaks within not-black text (e.g. comments (green here))
  showspaces=false,                % show spaces everywhere adding particular underscores; it overrides 'showstringspaces'
  showstringspaces=false,          % underline spaces within strings only
  showtabs=false,                  % show tabs within strings adding particular underscores
  stepnumber=2,                    % the step between two line-numbers. If it's 1, each line will be numbered
  stringstyle=\color{purple},      % string literal style
  tabsize=2,	                     % sets default tabsize to 2 spaces
  title=\lstname                   % show the filename of files included with \lstinputlisting; also try caption instead of title
}

\begin{document}
\author{Isaac Gibert}
\author{Ian Martorell}
\author{Sara Piñeiro}
\author{Esteban Ruiz}
\author{Eduard Sulé}
\affil{220024 - Aerodynamics, UPC ESEIAAT}
\title{Homework 2: Horseshoe Vortex Method}
\date{Dated: January 2016}
\maketitle

\section{Efecto del alargamiento en el $c_{L,\alpha}$}

\begin{figure}[H]
  \begin{center}
    \begin{tikzpicture}
        \begin{axis}[
            width=0.9\textwidth,
            height=0.4\textwidth,
            xlabel={Alargamiento},
            ylabel={$c_{L,\alpha}$},
            grid=major
            ]
            \addplot+[smooth,mark=*] table [
                x index=0, y index=1, col sep=comma, header=false
                ]{data/hw2_1.csv};
            \label{plot:hw2_1}
        \end{axis}
    \end{tikzpicture}
    \caption{Caption}
    \label{fig:hw2_1}
  \end{center}
\end{figure}

\section{Efecto de la flecha en el $c_{L,\alpha}$ y el $x_{ac}$}

\begin{figure}[H]
  \begin{tikzpicture}
      \begin{axis}[
          width=0.85\textwidth,
          height=0.4\textwidth,
          axis y line*=left,
          xlabel={Flecha},
          ylabel={$c_{L,\alpha}$},
          grid=major
          ]
          \addplot+[smooth,mark=*] table [
              x index=0, y index=1, col sep=comma, header=false
              ]{data/hw2_2.csv};
          \label{plot:hw2_2_cLalpha}
      \end{axis}
      \begin{axis}[
          width=0.85\textwidth,
          height=0.4\textwidth,
          axis y line*=right,
          axis x line=none,
          xlabel={Flecha},
          ylabel={$x_{ac}$},
          grid=major
          ]
          \addlegendimage{/pgfplots/refstyle=plot:hw2_2_cLalpha}\addlegendentry{$c_{L,\alpha}$}
          \addlegendimage{/pgfplots/refstyle=plot:hw2_2_xac}\addlegendentry{$x_{ac}$}
          \addplot+[smooth,mark=*] table [
              x index=0, y index=2, col sep=comma, header=false
              ]{data/hw2_2.csv};
          \label{plot:hw2_2_xac}
      \end{axis}
  \end{tikzpicture}
  \caption{Caption}
  \label{fig:hw2_2}
\end{figure}

\section{Efecto del estrechamiento en la distribución de sustentación local}

\begin{figure}[H]
  \begin{tikzpicture}
      \begin{axis}[
          width=0.85\textwidth,
          height=0.4\textwidth,
          xlabel={Cuerda~[\%]},
          ylabel={$c_{L}$},
          grid=major
          ]
          \addplot+[smooth,mark=none] table [
              x index=0, y index=1, col sep=comma, header=has colnames
              ]{data/hw2_3.csv};
          \label{plot:hw2_3_0.1}
          \addplot+[smooth,mark=none] table [
              x index=0, y index=2, col sep=comma, header=has colnames
              ]{data/hw2_3.csv};
          \label{plot:hw2_3_0.2}
          \addplot+[smooth,mark=none] table [
              x index=0, y index=3, col sep=comma, header=has colnames
              ]{data/hw2_3.csv};
          \label{plot:hw2_3_0.3}
          \addplot+[smooth,mark=none] table [
              x index=0, y index=4, col sep=comma, header=has colnames
              ]{data/hw2_3.csv};
          \label{plot:hw2_3_0.4}
          \addplot+[smooth,mark=none] table [
              x index=0, y index=5, col sep=comma, header=has colnames
              ]{data/hw2_3.csv};
          \label{plot:hw2_3_0.5}
          \addplot+[smooth,mark=none] table [
              x index=0, y index=6, col sep=comma, header=has colnames
              ]{data/hw2_3.csv};
          \label{plot:hw2_3_0.6}
          \addplot+[smooth,mark=none] table [
              x index=0, y index=7, col sep=comma, header=has colnames
              ]{data/hw2_3.csv};
          \label{plot:hw2_3_0.7}
          \addplot+[smooth,mark=none] table [
              x index=0, y index=8, col sep=comma, header=has colnames
              ]{data/hw2_3.csv};
          \label{plot:hw2_3_0.8}
          \addplot+[smooth,mark=none] table [
              x index=0, y index=9, col sep=comma, header=has colnames
              ]{data/hw2_3.csv};
          \label{plot:hw2_3_0.9}
          \addplot+[smooth,mark=none] table [
              x index=0, y index=10, col sep=comma, header=has colnames
              ]{data/hw2_3.csv};
          \label{plot:hw2_3_1}
          \addlegendimage{/pgfplots/refstyle=plot:hw2_3_0.1}\addlegendentry{$\lambda=0.1$}
          \addlegendimage{/pgfplots/refstyle=plot:hw2_3_0.2}\addlegendentry{$\lambda=0.2$}
          \addlegendimage{/pgfplots/refstyle=plot:hw2_3_0.3}\addlegendentry{$\lambda=0.3$}
          \addlegendimage{/pgfplots/refstyle=plot:hw2_3_0.4}\addlegendentry{$\lambda=0.4$}
          \addlegendimage{/pgfplots/refstyle=plot:hw2_3_0.5}\addlegendentry{$\lambda=0.5$}
          \addlegendimage{/pgfplots/refstyle=plot:hw2_3_0.6}\addlegendentry{$\lambda=0.6$}
          \addlegendimage{/pgfplots/refstyle=plot:hw2_3_0.7}\addlegendentry{$\lambda=0.7$}
          \addlegendimage{/pgfplots/refstyle=plot:hw2_3_0.8}\addlegendentry{$\lambda=0.8$}
          \addlegendimage{/pgfplots/refstyle=plot:hw2_3_0.9}\addlegendentry{$\lambda=0.9$}
          \addlegendimage{/pgfplots/refstyle=plot:hw2_3_1}\addlegendentry{$\lambda=1.0$}
      \end{axis}
  \end{tikzpicture}
  \caption{Caption}
  \label{fig:hw2_3}
\end{figure}

\section{Efecto del estrechamiento y la flecha en el factor de eficiencia de Oswald}

\section{Apéndice: Código}

\lstinputlisting[language=Matlab]{wing_discretization.m}
\lstinputlisting[language=Matlab]{rectangular_horseshoe.m}
\lstinputlisting[language=Matlab]{compute_induced_velocity.m}
\lstinputlisting[language=Matlab]{quarter_chord_sweep.m}
\lstinputlisting[language=Matlab]{HVM.m}
\lstinputlisting[language=Matlab]{hw2_1.m}
\lstinputlisting[language=Matlab]{hw2_2.m}
\lstinputlisting[language=Matlab]{hw2_3.m}
\lstinputlisting[language=Matlab]{hw2_4_aspect.m}
\lstinputlisting[language=Matlab]{hw2_4_sweep.m}
\lstinputlisting[language=Matlab]{shia_lebouf.m}

\end{document}
